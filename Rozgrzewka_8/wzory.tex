\documentclass[12pt,a4paper]{article}
\usepackage{polski}
\usepackage[utf8]{inputenc}
\usepackage{amsmath}
\usepackage{amsfonts}
\usepackage{amssymb}
\usepackage{graphicx}
\usepackage{lmodern}
\usepackage[left=2cm,right=2cm,top=2cm,bottom=2cm]{geometry}
\author{Patryk Lisik}
\title{IAD-wzory}
\renewcommand*\descriptionlabel[1]{\hspace\leftmargin$#1$}
\begin{document}
\maketitle
\begin{equation}\label{eq_1}
f(x)=w_0+\sum^c_{i=1}w_ik(||x-c_i||,\sigma_i )
\end{equation}
\begin{description}
\item[w_i] jest i-tą wagą
\item[c_i] jest i-tym centrum
\end{description}

\begin{equation}\label{eq_2}
k(d,\sigma)=\frac{1}{\sqrt{2\pi}\cdot\sigma }\cdot \exp\left( -\frac{d^2}{2\sigma ^2}\right) 
\end{equation}

\begin{equation}\label{eq_3}
Q(P)=\frac{1}{2P} \sum^P_{i=0} \left(f(x_i)-y_i\right) ^2
\end{equation}

\begin{equation}\label{eq_4}
w_n(t+1)=w_n(t)-\alpha \frac{\partial}{\partial w_n}Q(w(t))
\end{equation}

\section{Metoda najmniejszego spadku dla wag}

$$
\frac{\partial}{\partial w_n}Q(P)=\frac{\partial}{\partial w_n}\frac{1}{2P} \sum^P_{i=0} \left(w_0+\sum^c_{j=1}w_ik(||x_i-c_j||,\sigma_j )-y_i\right) ^2
$$

ze wzoru na pochodną funkcji złożonej $\left(f(g(x))\right)'=f'(g(x))g'(x)$
dla $n\neq 0$
$$
\frac{\partial}{\partial w_n}Q(P)=\frac{1}{P} \sum^P_{i=0} \left(w_0+\sum^c_{j=1}w_ik(||x_i-c_j||,\sigma_j )-y_i\right)\cdot w_n
$$
zatem

\begin{gather*}   
\frac{\partial}{\partial w_n}Q(P)=\\
\begin{cases}
  w_n(t+1)=w_n(t)-\alpha \frac{1}{P} \sum^P_{i=0} \left(w_0+\sum^c_{j=1}			w_ik(||x_i-c_j||,\sigma_j )-y_i\right)\cdot w_n & \text{dla } 
  n\in\{1,2\ldots j\}\\      
  w_n(t+1)=w_n(t)-\alpha \frac{1}{P} \sum^P_{i=0} \left(w_0+\sum^c_{j=1}			w_ik(||x_i-c_j||,\sigma_j )-y_i\right)  & \text{dla } n=0  
\end{cases}
\end{gather*}
\section{Metoda najmniejszego spadku dla promienia sąsiedztwa}
\begin{equation}
\sigma_n(t+1)=\sigma_n(t)-\alpha \frac{\partial}{\partial \sigma_n}Q(\sigma(t))
\end{equation}

\begin{equation*}
\begin{aligned}
\frac{\partial}{\partial \sigma_n}Q(P)=
\frac{\partial}{\partial \sigma_n}\frac{1}{2P} \sum^P_{i=0}
	\left(w_0+\sum^c_{j=1}w_ik(||x_i-c_j||,\sigma_j )-y_i\right) ^2= \\
\frac{1}{P} \sum^P_{i=0} \left(w_0+\sum^c_{j=1} \left( w_ik(||x_i-c_j||,\sigma_j )-y_i \right)\cdot\frac{\partial}{\partial \sigma_n}k(||x_i-c_j||,\sigma_j ) \right)
\end{aligned}
\end{equation*}

$$
\frac{\partial}{\partial \sigma}k(d,\sigma)=\frac{\partial}{\partial \sigma}\frac{1}{\sqrt{2\pi}\cdot\sigma }\cdot \exp\left( -\frac{d^2}{2\sigma ^2}\right) =
$$
\\
\\
ze wzoru na iloczyn funkcji $(f(x)g(x))'=f'(x)g(x)+f(x)g'(x)$

\begin{equation*}
\begin{aligned}
\frac{\partial}{\partial \sigma}k(d,\sigma)=\frac{1}{\sqrt{2\pi}}\left(-\sigma^{-2}\right)\exp\left( -\frac{d^2}{2\sigma ^2}\right) + \frac{1}{\sqrt{2\pi}\cdot\sigma } \exp\left( -\frac{d^2}{2\sigma ^2}\right) (-2\sigma)^{-3}=\\
-\frac{1}{\sqrt{2\pi}\cdot\sigma }\cdot \exp\left( -\frac{d^2}{2\sigma ^2}\right)\cdot \left(\frac{1}{\sqrt{2\pi}\cdot\sigma} +(2\sigma)^{-3} \right)=
-k(d,\sigma)\left(\frac{1}{\sqrt{2\pi}\cdot\sigma} +(2\sigma)^{-3} \right)
\end{aligned}
\end{equation*}
stąd
$$
\frac{\partial}{\partial \sigma_n}Q(P)=\frac{1}{P} \sum^P_{i=0} \left(\left[ w_0-\sum^c_{j=1} \left( w_ik(||x_i-c_j||,\sigma_j )-y_i\right) \right]\left[ k(||x_i-c_n||,\sigma_n )\left(\frac{1}{\sqrt{2\pi}\cdot\sigma_n} +(2\sigma_n)^{-3} \right)\right] \right)
$$
\end{document}